\section{\uppercase{BIM and Design for Deconstruction}}
\label{sec:introduction}


La  \textit{gestione dei rifiuti} \`e un tema che negli ultimi decenni si \`e fatto sempre pi\`u  attuale, considerando gli impatti economici, ambientali e energetici che da questa derivano.
 
Attraverso diverse Direttive che si sono susseguite nel corso degli ultimi anni (98/2008 EU, 1357/2014, …), la Comunit\`a Europea ha definito norme e obiettivi (qualitativi e quantitativi) che gli Stati Membri devono rispettare e ottenere - attraverso l’emanazione di regolamenti nazionali e la definizione di strumenti economici.
 
Lo smaltimento dei rifiuti, e dei costi diretti e indiretti che ne derivano, non \`e ovviamente estraneo nell’ambito delle attività delle Costruzioni e delle Demolizioni.
  
Nella Classificazione Europea dei Rifiuti (CER) [955/2014 EU] [in inglese European Waste Code/Classification], i rifiuti che si producono in particolare dalle attivit\`a di demolizione di edifici e fabbricati, sono identificati da una specifica classe - la \textit{17 - Construction And Demolition Wastes (Including Excavated Soil From Contaminated Sites)}.
 
Questa classificazione rende possibile, allo stesso tempo, sia una corretta identificazione del rifiuto prodotto dalle specifiche modalità di demolizione adottabili sia delle azioni di \textit{gestione del rifiuto} in termini di possibile \textit{riuso, riciclo o di smaltimento in discarica}.

Ci\`o detto, il quadro normativo che si determina nel contesto della gestione dei rifiuti, in particolare nei singoli contesti nazionali, risulta tutt’altro che chiaro.

Malgrado i costi derivanti dalla scelta di “smaltimento in discarica”, gli operatori del settore preferiscono questa strada piuttosto che rischiare di incorrere in sanzioni amministrative, o anche penali, per una mancata conformit\`a a regole e norme poco chiare.
 
Per superare questo approccio non risulta certo utile, nella sostanza, imporre d’ufficio delle soglie di minime di “gestione dei rifiuti” dalle attività di cantiere,  che pure sono parte essenziale delle direttive Europee citate.

E` all’interno di questo quadro normativo che rischia di essere \textit{farraginoso} che riteniamo utile promuovere l’utilizzo di strumenti (di supporto) che, grazie una modellazione 3D dell’edificio capace di integrare anche una descrizione semantica delle parti che lo compongono e delle loro relazioni, possano favorire il superamento delle difficolt\`a amministrative come pure l’efficacia nella corretta identificazione dei rifiuti prodotti, e quindi delle opportune scelte fra le possibili azioni \textit{virtuose}, ovvero quelle di recupero e di riuso.

A cui si aggiunge, in contesto di \textit{modellazione}, la possibilit\`a di identificare in modo puntuale i costi, così come i possibili ricavi derivanti dal percorso alternativo di \textit{riciclo/riuso} rispetto allo smaltimento, oltre che la composizione e integrazione di informazioni utili alla pianificazione delle attività di cantiere.
 
Inoltre, in un \textit{ambiente di modellazione e calcolo}, si rende possibile sia la verifica del raggiungimento delle soglie di \textit{riuso/recupero} previste dalle normative (una sorta di \textit{validazione dell’intervento sul cantiere}) che anche la possibilit\`a di confrontare fra loro diverse opzioni  





*** Spunti da sviluppare (?)

Uno strumento di editing ---veloce--- rende percorribile la strada di adempiere agli obblighi di legge

Dall’abstract di questo articolo Building Information Modeling (BIM) for existing buildings — Literature review and future needs (\url{www.sciencedirect.
com/science/article/pii/S092658051300191X }) 

mi sembra interessante la seguente lista di cose 

Results show scarce BIM implementation in existing buildings yet, due to challenges of 

(1) high modeling/conversion effort from captured building data into semantic BIM objects, 

(2) updating of information in BIM and 

(3) handling of uncertain data, objects and relations in BIM occurring in existing buildings.


dati circa lo elevato costo sociale, ambientale e economico delle materie prime utilizzate in edilizia .

il 10-15\% consumi energetici del settore C\&D si deve alla estrazione delle materie prime (Rapporto United Nations Environment Programme: \textit{Buildings and Climate Change} 2002)

il 25\% dei rifiuti prodotti (in Italia) deriva dal settore C\&D

