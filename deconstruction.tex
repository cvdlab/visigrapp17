\section{\uppercase{BIM and Design for Deconstruction}}
\label{sec:deconstruction}


\subsection{Regulatory framework}

Waste management is an issue that in recent decades has become increasingly more important, considering its economic, environmental and energy  impact.
The European Community has defined standards (EU 98/2008, 1357/2014) and goals (qualitative and quantitative) that Member States must comply with through the enactment of national regulations and the definition of economic instruments.

Waste disposal, with direct and indirect costs involved, is enforced by Construction and Demolition activities. In European Waste Code/Classification (CER) [955/2014 EU] waste from the demolition of buildings are identified as class 17 -- Construction And Demolition Wastes (including Excavated Soil From Contaminated Sites). 
This classification allows a correct identification of both the waste generated from adoptable demolition modes, and the waste produced by actions of re-use, recycling or landfilling.
 
The regulatory framework for waste management is far from clear, in particular in the national context. Despite the cost of the "landfill", this is often preferred rather than risking administrative or criminal penalties for failure to comply with unclear rules.  


\subsection{Proposed approach}




E` all’interno di questo quadro normativo che rischia di essere farraginoso che riteniamo utile promuovere l’utilizzo di strumenti (di supporto) che, grazie una modellazione 3D dell’edificio capace di integrare anche una descrizione semantica delle parti che lo compongono e delle loro relazioni, possano favorire il superamento delle difficoltà amministrative come pure l’efficacia nella corretta identificazione dei rifiuti prodotti, e quindi delle opportune scelte fra le possibili azioni virtuose, ovvero quelle di recupero e di riuso.

A cui si aggiunge, in contesto di modellazione, la possibilità di identificare in modo puntuale i costi, così come i possibili ricavi derivanti dal percorso alternativo di riciclo/riuso rispetto allo smaltimento, oltre che la composizione e integrazione di informazioni utili alla pianificazione delle attività di cantiere.
 
Inoltre, in un ambiente di modellazione e calcolo, si rende possibile sia la verifica del raggiungimento delle soglie di riuso/recupero previste dalle normative (una sorta di validazione dell’intervento sul cantiere) che anche la possibilità di confrontare fra loro diverse opzioni  





*** Spunti da sviluppare (?)

Uno strumento di editing ---veloce--- rende percorribile la strada di adempiere agli obblighi di legge

Dall’abstract di questo articolo Building Information Modeling (BIM) for existing buildings — Literature review and future needs (\url{www.sciencedirect.
com/science/article/pii/S092658051300191X }) 

mi sembra interessante la seguente lista di cose 

Results show scarce BIM implementation in existing buildings yet, due to challenges of 

(1) high modeling/conversion effort from captured building data into semantic BIM objects, 

(2) updating of information in BIM and 

(3) handling of uncertain data, objects and relations in BIM occurring in existing buildings.


dati circa lo elevato costo sociale, ambientale e economico delle materie prime utilizzate in edilizia .

il 10-15\% consumi energetici del settore C\&D si deve alla estrazione delle materie prime (Rapporto United Nations Environment Programme: Buildings and Climate Change 2002)

il 25\% dei rifiuti prodotti (in Italia) deriva dal settore C\&D

