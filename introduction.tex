%\vspace{-2mm}
\section{\uppercase{Introduction}}
\label{sec:introduction}
%\vspace{-2mm}

\noindent
A tendency to move away from the humanization of territories and to reuse already  built accommodation or accommodation which has fallen into disuse has become a pressing need in advanced societies. We have to integrate the ``zero energy'' model (each building has to produce the same amount of energy that it consumes) with the ``zero waste'' model , i.e. a new design paradigm where the waste materials from demolition become resources for reconstruction~\cite{altamura:12}. Building processes and designs have to be renewed to take account of environmental concerns. 

To reduce the impact of construction projects on the environment, the design needs to take the issue of building materials into consideration. Public administrations need suitable tools for the calculation and the control of reused or disposed materials. The new tools should handle the digital processing of materials throughout the project, supporting new project requirements such as: Design for Deconstruction, Design for Recycling and Design for Waste. In particular, a building life cycle, underpinned by a construction process which envisages cycles aligned to natural phenomena is the focus of this paper. 

In this work\footnote{\acks} we propose solutions that serve to close the circle of the building life-cycle, moving away from a traditional linear response with excessively high consumption energy rates (cradle to grave) and towards the reuse of materials in deconstruction/reconstruction (cradle to cradle), supported by computer aided selective demolition process.

All restructuring cycles should envisage de-construction and re-construction steps targeted towards the replacement of materials in order to achieve greater efficiency. The handling of these materials would require appropriate encoding both for the disposal (ECW - European Coding of Waste) and for the planning and design of new buildings (BIM --- Building Information Modeling). For this reason we need geo-referenced scenes of augmented reality based on fast, easily navigable and measurable 3D models. 

We already have excellent knowledge about construction costs (from scratch) but little is known about replacement rates (complete selective demolition). A modern selective demolition process requires human intervention, with high insurance costs due to the danger involved for those working in these activities. This latter point demands an alternative to human effort in these process. We suggest that automated robots could replace human effort; drones could operate in a semantically familiar context and give real-time updates as the reality contextually changes. 

We believe, therefore, that there is a big need for modern and easy-to-use modeling frameworks for building deconstruction in the AEC industry, to enable an augmented reality through semantic recognition by computer vision and by photogrammetric precision up to centimetric definition. Such virtual/augmented reality tools require both fast 3D building modeling and augmentation with semantic content, in order to be controlled in almost real time: this real challenge is also required by the future development of IoT. 	  
 
In this section we have discussed the motivation of the project described in this paper. The remaining sections are organized as follows.
Section~\ref{sec:deconstruction} introduces a more technical viewpoint about the state of deconstruction topics in Europe and in Italy.
Section~\ref{sec:application} describes the client application and the proposed workflow for quantity surveyors.
Section~\ref{sec:architecture} illustrates the framework architecture. 
Section~\ref{sec:modeling} shortly recalls the methodology, programming style and environment of our geometric computing approach to solid modeling.
In the conclusion section we outline the work to be done and provide our forecast about possible developments.
