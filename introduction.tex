\section{\uppercase{Introduction}}
\label{sec:introduction}

This project aims to contribute to a collective response to combatting climate change that afflict our times. In particular, a sensible contribution can come from a change of the \emph{building life-cycle}, regarded as a ``building organism'' that can be brought back in line with the cyclical nature of natural phenomena. We need solutions that serve to \emph{close the circle} of the building life-cycle, changing the traditional linear response (cradle to grave), greatly energy-consuming, toward cycles of materials reuse in deconstruction/reconstruction (cradle to cradle).
 
The tendency to not humanise new territory, but to better reuse as built and in disuse, is a compelling necessity in advanced societies. You must integrate the 'zero energy' model (each building has to produce the energy it consumes) with the 'zero waste' model, i.e. a new design paradigm where the waste materials from demolition become resources for reconstruction~\cite{altamura:12}. 

Building process and design must be renewed to accommodate environmental concerns. To reduce the environmental impact of construction projects, the design has to deal with the issue of materials, very  important concern for the governance of incentive policies for re-use. Public administrations need suitable tools for calculation and control of reused or disposed materials. The new instruments should handle the digital processing of materials in time, supporting new project requirements such as: Design for Deconstruction, Design for Recycling and Design for Waste.

All restructuring cycles should provide steps of deconstruction and re-construction targeted to the replacement of materials for the achievement of states of greater efficiency.
Consideration of materials would require appropriate encoding both for disposal (CER -- European Coding of Waste) and for the planning and design of the new (BIM -- Building Information Modeling). For this purpose we need geo-referenced scenes of augmented reality based on  fast, easily navigable and measurable 3D modeling. 

We already have an excellent knowledge of construction costs (from scratch) but little is known of the replacement rates (complete selective demolition). A modern selective demolition requires human intervention, with higher insurance costs for the danger of such interventions. This aspect opens the theme of ``driving'' automated robots to replace humans.  In this scene it must be possible to drive robots without human presence (drones) to  operate in a semantically known context, and giving real-time updates while reality contextually changes.

Therefore we believe that modern and easy-to-use modeling frameworks for building deconstruction in AEC industry are strongly needed, possibly augmented through semantic recognition by computer vision and by photogrammetric precision until to centimetric definition. Such virtual/augmented reality tools require both fast 3D building modeling and augmentation with semantic content, in order to be controlled in almost realtime: a real challenge also required by the future developments of IoT.
 
In this section we have discussed the motivation of the project described in this paper. The remaining sections are organized as follows.
Section~\ref{sec:deconstruction} introduces a more technical viewpoint about the state of deconstruction topics in Europe and in Italy.
Section~\ref{sec:modeling} shorthy recalls the methodology and the programming style and environment of our geometric computing approach to virtual and augmented reality.
Section~\ref{sec:application} describes the client application and its typical use-case. 
Section~\ref{sec:architecture} illustrates the framework architecture. 
In the conclusion section we outline the work to be done and provide our forecast about possible developments.
